%%=============================================================================
%% Conclusie
%%=============================================================================

\chapter{Conclusie}
\label{ch:conclusie}

%% TODO: Trek een duidelijke conclusie, in de vorm van een antwoord op de
%% onderzoeksvra(a)g(en). Wat was jouw bijdrage aan het onderzoeksdomein en
%% hoe biedt dit meerwaarde aan het vakgebied/doelgroep? Reflecteer kritisch
%% over het resultaat. Had je deze uitkomst verwacht? Zijn er zaken die nog
%% niet duidelijk zijn? Heeft het onderzoek geleid tot nieuwe vragen die
%% uitnodigen tot verder onderzoek?
Tijdens het ontwikkelen van de frontend applicaties bleek al snel dat het manipuleren van het DOM geen makkelijke zaak is voor functioneel programmeren, aangezien dit allemaal neveneffecten zijn en dus ongewenst is binnen dit paradigma. Een sterk interactieve interface zoals die van de metronoom is ook zeer moeilijk om functioneel te programmeren omdat deze vaak het bijhouden van een bepaalde state vereisen. Het verwerken van lijsten met data daarentegen is dan weer een sterkte van functioneel programmeren dankzij de map en reduce functies. Deze sterkte komt ook opnieuw naar voor in de backend doordat deze kleine operaties kunnen gedefinieerd worden in compacte functies en zo kunnen hergebruikt worden in grotere functies. 

De Object-Oriented frontend met zijn MVC structuur blijft nog steeds robuust en duidelijk. De logica zit verstopt in de objecten wat het enorm makkelijk maakt om te gebruiken voor een interactieve user interface. Het enige wat er moet gebeuren is de View (HTML) linken met de juiste Controllers. Voor de metronoom was OOP de ideale oplossing aangezien elke actie werd gelinkt met een functionaliteit of het wijzigen van de state van het Metronome object. Het doorlopen van lijsten op de tradionele OOP manier is echter een minpunt maar dit kan eenvoudigweg vermeden worden door de combinatie met functioneel programmeren.

De backends tonen echter weinig verschillen. Bij FP wordt het afhandelen van de routes gedelegeerd aan functies die bestaan uit allerlei kleinere deelfuncties die alle nodige functionaliteit bevatten om te voldoen aan de binnenkomende requests. Dit hergebruik van code vermijdt copy/paste werk en zorgt ervoor dat het geheel ook leesbaarder is.

Het delegeren van het afhandelen van requests aan de Controllers zorgt ook voor een duidelijke en leesbare flow. Het enige nadeel is dat de implementatie van de methodes op de controllers vaak specifiek is (behalve bij abstracte klassen) en dus moeilijk om te hergebruiken.

Opvallend aan de uiteindelijke resultaten van het vergelijken van de applicaties is dat er op zich weinig verschillen zijn zowel op het vlak van lijnen code als performantie en complexiteit (in de mate dat dit gemeten kon worden). De ontwikkelde applicaties zijn echter klein qua scope, een onderzoek met applicaties van grotere scope zou een beter beeld kunnen opleveren van de verschillen in maatstaven van deze applicaties.

Om deze vergelijking zo eerlijk mogelijk te houden werden er geen externe libraries gebruikt die niet gebruikt werden door de andere applicatie. Dit heeft er echter voor gezorgd dat er minder puur FP of OOP kon gewerkt worden, zoals bv. de afwezigheid van access modifiers bij klassen om aan encapsulatie te doen of het definiëren van interfaces voor polymorfisme. Libraries zoals RxJS en Lodash of Ramda zijn een goede hulp bij het gebruik van het FP paradigma binnen JavaScript dankzij zaken zoals ingebouwde curry functionaliteit, het samenstellen van functies via een compose functie en het manipuleren van het DOM via Reactive programmeren. De applicaties werken ook enkel met puur JavaScript, iets wat niet veel meer voorkomt op professioneel niveau door de opkomst van vele populaire frameworks voor frontend. Bepaalde frameworks lenen zich beter tot OOP of FP wat dus zeker een rol speelt bij de keuze tussen beide paradigma's. Een vergelijkende studie per framework kan deze sterktes en zwaktes aantonen zodat er binnen die frameworks optimaal met OOP of FP kan gewerkt worden.

Functioneel programmeren kan veel voordelen bieden maar is toch net iets te omslachtig om mee te werken binnen een pure JavaScript omgeving zonder externe libraries, zoals te zien was bij het metronoom gedeelte. Er moeten veel toegevingen gedaan worden waarbij er wordt afgeweken van de FP principes, terwijl dit bij OOP niet het geval is. Het volgen van een pure OOP stijl binnen een pure JavaScript omgeving blijft nog steeds eenvoudiger. Het kiezen van paradigma zal dus vooral afhangen van het gekozen framework en het gebruik en kennis van externe libraries.

