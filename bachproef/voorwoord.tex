%%=============================================================================
%% Voorwoord
%%=============================================================================

\chapter*{Woord vooraf}
\label{ch:voorwoord}

%% TODO:
%% Het voorwoord is het enige deel van de bachelorproef waar je vanuit je
%% eigen standpunt (``ik-vorm'') mag schrijven. Je kan hier bv. motiveren
%% waarom jij het onderwerp wil bespreken.
%% Vergeet ook niet te bedanken wie je geholpen/gesteund/... heeft

Ik ben Sam Dhondt, student Toegepaste Informatica, en met deze bachelorproef hoopte ik mij wat meer te kunnen verdiepen in het functioneel programmeren. Ik ben de term voor het eerste tegengekomen tijdens de les Webapplicaties IV en dit bleek al iets te zijn waar ik mee in aanraking was gekomen in andere lessen. Ik hou ervan om nieuwe technieken, talen of frameworks te gebruiken bij het programmeren dus ik was zeer benieuwd naar de werking van functioneel programmeren. Op school ligt de focus eerder op Object-Geörienteerd programmeren, dit blijft dan ook een populair paradigma en is nog steeds gemakkelijk om mee aan de slag te gaan als beginnend programmeur. Toch wou ik mij verder verdiepen in deze voor mij nieuwe techniek. Echter kom je online vaak codevoorbeelden tegen van simpele lijntjes code die worden vergeleken maar vaak geen praktische toepassing hebben. Daarom leek de bachelorproef mij het ideale moment om een deftige vergelijking te kunnen maken tussen deze twee programmeerparadigma's. Functioneel programmeren is geen simpel concept om aan te leren door zijn abstractie en wiskundige achtergrond, maar toch hoop ik dat dit wat meer aan bod zal komen in de opleiding. Wie weet misschien met een pure FP taal zoals Haskell?

Verder wil ik nog mijn promotor, Stefaan Samyn, en co-promotor, Kristof Van Miegem, bedanken voor de begeleiding tijdens dit laatste onderdeel van de opleiding.