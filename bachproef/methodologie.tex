%%=============================================================================
%% Methodologie
%%=============================================================================

\chapter{Methodologie}
\label{ch:methodologie}

%% TODO: Hoe ben je te werk gegaan? Verdeel je onderzoek in grote fasen, en
%% licht in elke fase toe welke stappen je gevolgd hebt. Verantwoord waarom je
%% op deze manier te werk gegaan bent. Je moet kunnen aantonen dat je de best
%% mogelijke manier toegepast hebt om een antwoord te vinden op de
%% onderzoeksvraag.

\section{Voorbereiding}
Omdat JavaScript geen pure FP taal is en omdat het nodig is dat de concepten hiervan grondig gekend zijn werd eerst de FP taal Haskell onderzocht. Door deze taal meer uit te diepen werd er inzicht verkregen in de totale werking van een pure FP taal zodat dit optimaal kan worden toegepast op de applicatieontwikkeling binnen JavaScript.

\subsection{Verloop}

\section{Uitvoering}
De applicatieontwikkeling bestaat uit vier delen: twee applicaties, één met een OOP aanpak en de ander met een FP aanpak, elk met een frontend en backend gedeelte. Er werd gekozen om beide aspecten te ontwikkelen zodat er ook op dat niveau een vergelijking mogelijk is tussen frontend en backend in zowel OOP als FP. Frontend technieken met backend vergelijken is irrelevant en enkel één van beide vergelijken geeft een te oppervlakkig beeld van de voordelen en nadelen van beide paradigma's.

\subsection{Object-Oriented Frontend}
\subsection{Object-Oriented Backend}
\subsection{Functioneel Frontend}
\subsection{Functioneel Backend}